\chapter{Introduction}
\label{chap:Introduction}

\section{Purpose of this Document}
\label{sec:Purpose}

Instead of giving the students a lot of Powerpoint slides that compress the complex topics of this course into a bunch of bullet points, we wanted the students to have a dedicated document for preparing their exam. Unfortunately, such a document is a lot more work and it needs several iterations of thorough reviewing. Therefore, we came up with the compromise that the students write this document theirselves and the lecturers only have to review and improve it. 

Maybe you ask: "What is wrong with Powerpoint?" 

A Powerpoint presentation should aid a presenter in presenting a certain message or information to his audience. The slides alone often miss critical parts (that couldn't easily condensed into bullet points) and are unclear without the corresponding vocal line of the presenter. Nevertheless, students end up alone with the slides trying to prepare their examinations.

Apart from that, try to google topics like "disadvantages of powerpoint" or "why we should ban Powerpoint" and you will find interesting articles that connect Powerpoint and the accident of the Columbia Space Shuttle [\href{http://www.nytimes.com/2003/12/14/magazine/2003-the-3rd-annual-year-in-ideas-powerpoint-makes-you-dumb.html}{1}] [\href{http://www.edwardtufte.com/bboard/q-and-a-fetch-msg?msg_id=0001yB&topic_id=1&topic=Ask+E%2eT%2e}{2}].

\section{How to Contribute to this Document?}
\label{sec:Contribute}

Basically there are two aspects. At first you need to get the \LaTeX\  source code of this document from our \href{https://github.com/dasys-lab/service-robot-ai}{service-robot-ai GitHub Repository}. It is located in the folder \verb#doc/latex#. Anybody can download it, but for making changes to it you need to have an GitHub-Account which was given the corresponding privileges. That is typically arranged during the first session of this course. 

This leaves us with the requirements, that you are able to write \LaTeX\ and know how to use GIT in combination with GitHub. Regarding GIT and GitHub we recommend to read Section~\ref{sec:Git}. 

Maybe \LaTeX\ is hard to master when it comes to tables, graphics and customising the layout of a document, but you don't have to do that for contributing to this document. You should only be able to write text referencing images and place useful links from time to time. Furthermore, you should be able to structure the content into chapters, sections, and subsections. Everything just mentioned is already done in this document, so you can start by copying commands from the corresponding sections. 

For further reference and an in-depth study of \LaTeX, we recommend the following sources:

\begin{description}
  \item[\href{http://www.wiwiss.fu-berlin.de/fachbereich/vwl/iso/links/latex_einfuehrung_manuela_juergens.pdf}{\LaTeX\ \textendash\ eine Einf\"uhrung und ein bisschen mehr}:] A little bit longer introduction to \LaTeX\ basics.
  \item[\href{https://en.wikibooks.org/wiki/LaTeX}{\LaTeX\ WikiBook}:] Nice for looking up simple stuff. 
  \item[\href{https://tex.stackexchange.com}{tex.stackexchange.com}:] Stack Overflow for \LaTeX
  \item[\href{http://ftp.fau.de/ctan/graphics/pgf/base/doc/pgfmanual.pdf}{TikZ}:] \LaTeX package for designing advanced graphics directly in \LaTeX.
\end{description}

The final issue is to decide which editor you choose for compiling this \LaTeX document. We recommend \href{http://www.xm1math.net/texmaker/}{TexMaker}, because it is available for Ubuntu and Windows, although, there currently exists an issue with short cuts under Ubuntu 16.04 LTS. Consider this \href{http://askubuntu.com/questions/786280/texmaker-shortcuts-not-working-on-ubuntu-16-04}{Ask Ubuntu Post} for solving it.

With TexMaker you only have to open the \verb#Software_Reference_Book.tex# file and click on \verb#Quick Build#. If you get compile errors, you probably need to install the necessary \LaTeX packages. Under Ubuntu this is done by executing the following console command:

\verb#sudo apt install texlive-full#

\section{How to Setup your Development-PC?}
\label{sec:SetupPC}

The development environment recommended to develop software for the TurtleBots of the Distributed Systems Department is the result of best practises over years and continuously under development. Nevertheless, we try to keep this section up-to-date. There also exist scripts for doing these steps, however for the purpose of getting hands on experience we recommend to drudge through the drudgery at least once and thereby trying to understand what is actually going.

\begin{description}
	\item[1. Install Ubuntu] The currently used operating system is Ubuntu 18.04 LTS (long-term support) with all available updates installed:\\
	\verb#sudo apt update#\\
  \verb#sudo apt upgrade#
	\item[2. Install Ubuntu - Extra Packages] Simply install the package with the following command: \verb#sudo apt install g++ make binutils cmake libssl-dev#\\
	\verb#libboost-system-dev vim git ssh myrepos capnproto#\\
	\verb#libcapnp-dev libqt5x11extras5-dev qtscript5-dev liblua5.1-0-dev#\\
	\verb#bison re2c libcanberra-gtk-module:i386 libqt5webkit5-dev libsfml-dev#
	\item[3. Install ZMQ] Execute the following lines in a terminal:\\
\verb#sudo su#\\
\verb#echo "deb http://download.opensuse.org/repositories/network:/#\\\verb#   messaging:/zeromq:/git-draft/xUbuntu_18.04/ ./" >> /etc#\\\verb#   /apt/sources.list#\\
\verb#exit#\\
\verb#wget https://download.opensuse.org/repositories/network:/#\\\verb#   messaging:/zeromq:/git-draft/xUbuntu_18.04/Release#\\\verb#   .key -O- | sudo apt-key add#\\
\verb#sudo apt update#\\
\verb#sudo apt install libzmq3-dev#
	\item[4. Install Clingo] Clone the \href{https://github.com/potassco/clingo}{Clingo Repository} and follow their installation instructions:\\
		\verb#cd#\\
		\verb#git clone https://github.com/potassco/clingo#\\
  	\verb#cd clingo#\\
  	\verb#git submodule update --init --recursive#\\
		\verb#mkdir build#\\
  	\verb#cmake --build build --target install#\\
  	\verb#cd build#\\
  	\verb#cmake ..#\\
  	\verb#make#\\
  	\verb#sudo make install#\\
	\item[5. Install ROS Desktop] The currently used ROS version is ROS Melodic Morenia with all available updates installed. Like Ubuntu 18.04 it is also a long-term support version. The instructions for installing ROS under Ubuntu can be found on the corresponding \href{http://wiki.ros.org/melodic/Installation/Ubuntu}{ROS Wikipage}.
	\item[6. Install ROS - Extra Packages] Simply install the package(s) with the following command: \verb#sudo apt install python-catkin-tools#
	\item[7. Create ROS Workspace] We recommend to follow this folder structure, as things can get complicated otherwise. 
	\begin{itemize}
		\item \verb#cd#
		\item \verb#mkdir -p rosws/src#
		\item \verb#cd rosws#
		\item \verb#catkin init#
	\end{itemize}
	\item[8. Checkout Github Repositories] You need access to our GIT repositories on GITHub. Therefore, please provide your GITHub Username to us. We will grant you the corresponding permissions. Furthermore, it is recommended to setup ssh access to your GITHub account, in order to avoid typing the passwords over and over again:
	\begin{itemize}
		\item \verb#ssh-keygen# followed by pressing enter until the command terminates. ;-)
		\item Follow the \href{https://help.github.com/articles/adding-a-new-ssh-key-to-your-github-account/}{GITHub Tutorial about adding SSH-Keys to your account}.
	\end{itemize}
	The next steps are for checking out the repositories under the assumption that you have the corresponding permissions:
	\begin{itemize}
		\item \verb#cd ~/rosws/src#
		\item \verb#git clone git@github.com:dasys-lab/alica.git#
		\item \verb#git clone git@github.com:dasys-lab/alica-plan-designer-fx.git#
		\item \verb#git clone git@github.com:dasys-lab/alica-supplementary.git#
		\item \verb#git clone git@github.com:dasys-lab/essentials.git#
		\item \verb#git clone git@github.com:dasys-lab/aspsuite.git#
		\item \verb#git clone git@github.com:dasys-lab/capnzero.git#
		\item \verb#git clone git@github.com:dasys-lab/service-robot-ai.git#
		\item \verb#git clone git@github.com:dasys-lab/service-robot-grid-simulation.git#
		\item \verb#git clone git@github.com:dasys-lab/telegram-client.git#
		\item \verb#git clone git@github.com:dasys-lab/tgbot-cpp.git#
		\item \verb#git clone git@github.com:dasys-lab/tmxparser.git#
	\end{itemize}
	\item[9. Install TG-Bot] For the telegram-client repository to compile you need to install the tgbot-cpp executable (from the last repository in Step 8):
	\begin{itemize}
		\item \verb#cd ~/rosws/src/tgbot-cpp#
		\item \verb#cmake .#
		\item \verb#make#
		\item \verb#sudo make install#
	\end{itemize}
	\item[10. Install tmxparser] For the grid simulator we use the tmxparser to load worlds given in tmx format.
	\begin{itemize}
		\item \verb#cd ~/rosws/src/tmxparser#
		\item \verb#mkdir build#
		\item \verb#cd build#
		\item \verb#sudo cmake .. -DBUILD_TINYXML2=ON#
		\item \verb#sudo make#
		\item \verb#sudo make install#
	\end{itemize}
	\item[11. General Configurations] You need to configure some of the installed tools:
		\begin{itemize}
			\item \textbf{git:} Copy the 'gitconfig' file from the service-robot-ai/configuration folder into your home folder and rename it to '.gitconfig'. Open the '.gitconfig' file and replace the 'email' and the 'name' in the 'user' section.
			\item \textbf{myrepos:} Jump into each repository and execute: \verb#mr register#
			\item \textbf{.bashrc:} You need to enter the lines from Appendix~\ref{sec:bashrc} into your '.bashrc' file. Alternatively, you can append the content of the 'bashrc' file from the service-robot-ai/configuration folder at the end of your '.bashrc' file in your home folder.
			\item \textbf{mr branches:} Add the following lines at the beginning of the '.mrconfig' file in your home folder:\\
			\verb#[DEFAULT]#\\
			\verb#branch = git branch#
		\end{itemize}
	\item[12. Switch Repository Branches] During the lecture, we are working on the following branches of the corresponding GIT repositories:
	\begin{itemize}
		\item alica - newPD
		\item alica-plan-designer-fx - behaviourCodeGen
		\item alica-supplementary - CapnZero
		\item telegram-client - master
		\item essentials - CapnZero
		\item capnzero - master
		\item aspsuite - asp\_dev
		\item service-robot-ai - capnzero
		\item service-robot-grid-simulation - master
	\end{itemize}
	The main work will take place within service-robot-ai, supplementary, and service-robot-grid-simulation. That is why the master branch (incl. branch protection) for alica and clingo are ok for us. In order to switch a branch execute\\\verb#git checkout <branch name># within the repositories root folder.
	\item[13. Compile Workspace] Within the 'rosws/src' folder execute \verb#catkin build#.
	\item[14. Setup CLion] Download CLion from Jetbrain's \href{https://www.jetbrains.com/clion/download/}{Download Page}. Get a student license via this \href{https://www.jetbrains.com/student/}{application}. It works quite fast. In order to open any ros-package in CLion, just let CLion open the corresponding CMakeLists.txt file and choose to open it as project, if you are asked to.
\end{description}

\section{Unix Basics}
\label{sec:LinuxBasics}

Differences between Unix and Windows can be found in this \href{https://www.techrepublic.com/blog/10-things/10-fundamental-differences-between-linux-and-windows/}{nice article}. This \href{https://www.howtogeek.com/137096/6-ways-the-linux-file-system-is-different-from-the-windows-file-system/}{special article about the different filesystems} is also very enlighting to read for Unix newcomers.

In order to operate easily with a Unix-based operating system like Ubuntu, you should understand the following terminal commands including their common parameters 

\begin{description}
\item [cd] Changing the directory, including special folders like '.', '..', and '$\sim$'.
\item [mkdir] Creating folders, including '-p' for nested folders.
\item [rm] Deleting files and '-r' for folders.
\item [ls] Showing the content of the current folder.
\item [ll] Like ls, just including several more information like permissions etc.
\item [grep] Finding strings inside of files, incl. the meaning of each of the three letters '-inr': \verb#grep "test" .# 
\item [find] Finding files: \verb#find . -name "regex-expression"#
\item [vim] Terminal-based editor for quick editing. Useful commands ':wq', 'ESC', 'i'
\item [ssh] Connecting to another machine goes like: \verb#ssh -pPORT USER@IP#,\\e.\,g., \verb#ssh -p222 cn@141.51.122.141#
\item [ssh-keygen] Creates a new ssh-key (or overwrites the old one).
\end{description}

Here are some resources for your own studies. Nevertheless, our experience is that, if you are open and willing to learn, attending the lecture in an attentive way, will be enough to understand everything you need.
\begin{itemize}
\item \href{https://www.digitalocean.com/community/tutorials/an-introduction-to-linux-basics}{Basic Terminal Stuff}
\item \href{https://www.digitalocean.com/community/tutorials/linux-permissions-basics-and-how-to-use-umask-on-a-vps}{Linux Permission System}
\item \href{https://www.tutorialspoint.com/vim/vim_quick_guide.htm}{Vim - "Quick" Guide}
\end{itemize}

And finally always remember: You can 'google' everything you need to know very easily. So don't get frustrated, just keep digging.
